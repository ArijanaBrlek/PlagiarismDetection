% Paper template for TAR 2016
% (C) 2014 Jan Šnajder, Goran Glavaš, Domagoj Alagić, Mladen Karan
% TakeLab, FER

\documentclass[10pt, a4paper]{article}

\usepackage{tar2016}

\usepackage[utf8]{inputenc}
\usepackage[pdftex]{graphicx}
\usepackage{booktabs}
\usepackage{amsmath}
\usepackage{amssymb}

\title{Plagiarism detection}

\name{Arijana Brlek, Petra Franjić, Nino Uzelac} 

\address{
University of Zagreb, Faculty of Electrical Engineering and Computing\\
Unska 3, 10000 Zagreb, Croatia\\ 
\texttt{\{arijana.brlek, petra.franjic, nino.uzelac\}@fer.hr}\\
}
          
         
\abstract{ 
This document provides the instructions on formatting the TAR project paper in \LaTeX{}. This is where you write the abstract (i.e., summary) of the work you carried out. The abstract is a paragraph of text ranging between 70 and 150 words.
}

\begin{document}

\maketitleabstract

\section{Introduction}

\section{Related work}

\section{Dataset}

\section{Implementation}

\subsection{Preprocessing}

\subsection{Seeding}

\subsection{Extension}

\subsection{Filtering}

\section{Evaluation and results}

\section{Referencing literature}


References to other publications should be written in brackets with the last name of the first author and the year of publication, e.g., \citep{chomsky-73}.  Multiple references are written in sequence, one after another, separated by semicolon and without whitespaces in between, e.g., \citep{chomsky-73,chave-64,feigl-58}. References are typically written at the end of the sentence and necessarily before the sentence punctuation.

If the publication is authored by more than one author, only the name of the first author is written, after which abbreviation \emph{et al.}, meaning \emph{et alia}, i.e.,~and others is written as in \citep{johnson-etc}. If the publication is authored by only two authors, then the last names of both authors are written \citep{johnson-howells}.

If the name of the author is incorporated into the text of the sentence, it should not be in the brackets (only the year should be there). E.g.,~``\citet{chomsky-73}
suggested that \dots''. The difference is whether you reference the publication or the author who wrote it. 

The list of all literature references is given alphabetically at the end of the paper. The form of the reference depends on the type of the bibliographic unit: conference papers,
\citep{chave-64}, books \citep{butcher-81}, journal articles
\citep{howells-51}, doctoral dissertations \citep{croft-78}, and book chapters \citep{feigl-58}. 

All of this is automatically produced when using BibTeX. Insert all the BibTeX entries into the file \texttt{tar2016.bib}, and then reference them via their symbolic names.

\section{Conclusion}

Conclusion is the last enumerated section of the paper. It should not exceed half of a column and is typically split into 2--3 paragraphs. No new information should be presented in the conclusion; this section only summarizes and concludes the paper.

\bibliographystyle{tar2016}
\bibliography{tar2016} 

\end{document}

